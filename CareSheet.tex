%
%% Document Class (Koma Script) -----------------------------------------
%% Doc: scrguien.pdf
\documentclass[%
  %draft=true,     % draft mode (no images, layout errors shown)
  draft=false,     % final mode
  %%% --- Paper Settings ---
  paper=a4,
  paper=portrait, % landscape
  pagesize=auto, % driver
  %%% --- Base Font Size ---
  fontsize=12pt,%
  %%% --- Koma Script Version ---
  version=last, %
  %%% --- Global Package Options ---
  ngerman, % language (passed to babel and other packages)
  % parskip,  % use empty lines between paragraphs instead of indenting the first line
  numbers=noenddot,
  listof=totoc,        % add lists of figures and tables to Table of Contents
  bibliography=totoc,  % add cited works to Table of Contents
]{scrreprt} % Classes: scrartcl, scrreprt, scrbook\usepackage[ngerman]{babel}

\newif\iftitlepage{}
\newif\iftableofcontents{}
\newif\iflistoffigures{}
\newif\iflistoftables{}
\newif\iflistofformeln{}
\newif\ifglossary{}
\newif\ifbibliography{}

% include settings.tex
%%% general settings you might want to modify

\providecommand*{\Version}{1.0}
\providecommand*{\Title}{Animal Care Sheet}
\providecommand*{\Name}{Max Mustermann}

% PDF meta data
\providecommand*{\pdfsubject}{Short eabstract what this is about}
\providecommand*{\pdfkeywords}{care sheet, pets, cats, disgusting, feeding}


%%% general display settings

%% uncomment lines you want to use:
% \providecommand*{\prefbiblioname}{Literaturverzeichnis}
% \titlepagetrue{}          %% Titelseite
% \tableofcontentstrue{}    %% Inhaltsverzeichnis
% \listoffigurestrue{}      %% Abbildungsverzeichnis
% \listoftablestrue{}       %% Tabellenverzeichnis
% \glossarytrue{}           %% Glossar
% \listofformelntrue{}      %% Formelverzeichnis
% \appendixtrue{}           %% Anhang
% \bibliographytrue{}       %% Literaturverzeichnis


% include packages and other general preamble
% dummy comment for file-wide intellisense errors

%\usepackage{graphicx} % enables use of eps graphics (encapsulated PostScript). Activate if needed.
%\usepackage{newtx} % replacement of previously used "times" package (using Times font as default)
\usepackage{babel}
\usepackage{supertabular}
\usepackage{wrapfig}
\usepackage{multirow}
\usepackage[onehalfspacing]{setspace}
\usepackage{scrhack}  % fix float warning of KOMA produced when including listings
\usepackage{listings}
\usepackage{mathptmx}
\usepackage{geometry}
\usepackage{helvet}
\usepackage{courier}
\usepackage{setspace}
\usepackage{textcomp}
\usepackage[T1]{fontenc}
\usepackage[utf8]{inputenc}
\usepackage{float} % Notwendig fuer figure[h]
\usepackage[german=quotes]{csquotes}
\usepackage[style=alphabetic]{biblatex} % alternative for sort: iso-authoryear
\usepackage{xurl} % better line-breaking than url package. Needs to be added after biblatex to work in bibliography.
\usepackage{pdfpages}
\usepackage{calc} % for calculations with text width
% Fuer Schriftart Arial
%\usepackage[scaled]{uarial}

% Installation der Arial Schriftart unter Linux.
% wget http://tug.org/fonts/getnonfreefonts/install-getnonfreefonts
% texlua install-getnonfreefonts
% getnonfreefonts -r
% getnonfreefonts arial-urw


% PDF Einstellungen für Verlinkungen
\usepackage[
  pdftitle={\Title},
  pdfsubject={\pdfsubject},
  pdfauthor={\Name},
  pdfkeywords={\pdfkeywords}
  hyperfootnotes=false,
  colorlinks=true,
  linkcolor=black,
  urlcolor=black,
  citecolor=black
]{hyperref}

\ifglossary{}
  %%% Abkürzungsverzeichnis (Glossar) Neues Paket (kann nomencl und acronym ersetzen)
  % muss nach hyperref eingebunden werden, um das Paket zu nutzen
  % Abkürzungen werden nur im Glossar angezeigt, wenn sie im Dokument mindestens einmal genutzt wurden
  \usepackage[
    % style=long,
    abbreviations, % Setzt Abkürzungen in ein gesondertes Verzeichnis (nur wenn Glossarverzeichnis auch angezeigt wird)
    % footnote, % Setzt eine Fußnote beim ersten verwendet wird
    % nomain,
    % style=altlist,
  ]{glossaries-extra}
  \setglossarystyle{super}
  \makeglossaries{} % Glossar generieren
\fi

\newfloat{Formel}{H}{for}

%% FAQ environment
\newenvironment{faq}{}{}
\DeclareSectionCommand[
  runin=false,                                        % start the answer in a new line
  afterskip=0.25\baselineskip plus -1ex minus -.2ex,  % chktex 1 commands cannot be terminated with curly braces in arguments
  beforeskip=-2.5ex plus -1ex minus -.2ex,
  indent=0pt,
  level=4,
  font=\usekomafont{paragraph}\itshape, %% using the same font as paragraph, but italic
  tocindent=10em,
  tocnumwidth=5em,
  counterwithin=subsubsection,
  style=section,
]{question}
\newcommand{\faqitem}[2]{\question{#1}{\setlength{\leftskip}{\parindent}#2\par}}
\def\questionautorefname{Frage} % for referring to labels within a question with \autoref, though I prefer referencing with the custom \faqref command
\newcommand*{\faqref}[1]{\hyperref[{#1}]{\appendixautorefname{}~\ref*{#1}, \questionautorefname{}~``\nameref*{#1}''}}


\renewcommand\UrlFont{\color{black}\rmfamily\itshape} % chktex 6 ignore missing '\/' after \itshape

\renewcommand{\familydefault}{\rmdefault}
\newcommand{\bflabel}[1]{\normalfont{\normalsize{#1}}\hfill}


% auxiliary commands
\input{preamble/commands}

% Style settings
%% Für Codeblöcke mit Syntax-Highlighting
%% http://www.ctan.org/tex-archive/macros/latex/contrib/minted/
%% Einkommentieren fuer Minted Syntax Highlighting
%\usepackage{minted}
%\definecolor{bg}{rgb}{0.95,0.95,0.95}

\makeatother

\geometry{a4paper, left=45mm, right=20mm, top=30mm, bottom=30mm, head=21.74998pt}
\setlength{\footheight}{21.74998pt}

\renewcommand*{\chapterheadstartvskip}{\vspace*{0\baselineskip}}% Abstand einstellen

\pagenumbering{roman}


\usepackage[automark,headsepline]{scrlayer-scrpage}

\clearpairofpagestyles{}
\ifoot[Version \Version{}]{Version \Version{}}
\cfoot[\pagemark{}]{\pagemark{}}
\ofoot[\today{}]{ \today{}}
\lehead{\headmark{}}
\rohead{\headmark{}}

\pagestyle{scrheadings}

\ifglossary{}
  \input{content/glossar}
\fi

% Literature, if you ant to quote any (because why not?)
\addbibresource{literature.bib}

% BEGIN DOCUMENT %%%%%%%%%%%%%%%%%%%%%%%%%%%%%%%%%%%%%%%%%%%%%%%%%%%%%%%%%%%%%%%
\begin{document}
%%%%%%%%%%%%%%%%%%%%%%%%%%%%%%%%%%%%%%%%%%%%%%%%%%%%%%%%%%%%%%%%%%%%%%%%%%%%%%%%

% TITEL PAGE %%%%%%%%%%%%%%%%%%%%%%%%%%%%%%%%%%%%%%%%%%%%%%%%%%%%%%%%%%%%%%%%%%%

\iftitlepage{}
  \begin{titlepage}
  \newgeometry{left=25mm, right=20mm, top=35mm, bottom=30mm}
  \begin{center}
    \thispagestyle{empty}

    \Large{\textbf{\Title}}

    \vfill
    \onehalfspacing{}
    \normalsize

    Version 0.9 from \today

  \end{center}

  \restoregeometry{}
\end{titlepage}
\fi

%%%%%%%%%%%%%%%%%%%%%%%%%%%%%%%%%%%%%%%%%%%%%%%%%%%%%%%%%%%%%%%%%%%%%%%%%%%%%%%%

\normalsize

% start with a checklist for each visit
\begin{spacing}{1.5} % main content gets 1.5 line spacing
  \begin{center}
  \Huge{\textbf{Checkliste Woche \makebox[3cm]{\hrulefill}}}
\end{center}
Diese Seite enthält eine Checkliste, was bei einem Besuch zu tun ist.
Detailliertere Informationen zu den Tieren und ihrer Pflege folgen auf den nächsten Seiten.

% \begin{multicols}{2}
\section*{2--3x pro Woche}
\begin{todolistx3}
  \item Wasserperlen der Schaben prüfen
  \item \hyperref[sec:Ameisen_sub:Zucker]{Zuckerlösung der Ameisen prüfen}
  \item Überreste von Lebendfutter aus Ameisenarenen entfernen
  \item Wasser der Ameisen prüfen
  \item
  \item
  \item
  \item \textit{optional:} Schaben füttern (frisches Gemüse/Früchte)\\und Futterreste entfernen
\end{todolistx3}

\section*{1x pro Woche}
\begin{todolist}
  \item Ameisen: Luftbeton-Nest befeuchten
  \item Schaben füttern (Trockenfutter auffüllen)
  \item Ameisen füttern (Lebendfutter oder Pulver)
  \item
  \item
  \item
  \item Luftbeton-Nest nochmals befeuchten
\end{todolist}
% \end{multicols}

\end{spacing}


\begin{spacing}{1.0} % single-line spacing for content tables

  % Inhaltsverzeichnis %%%%%%%%%%%%%%%%%%%%%%
  \iftableofcontents{}
    \tableofcontents
  \fi

  % Abbildungsverzeichnis %%%%%%%%%%%%%%%%%%%%%%
  \iflistoffigures{}
    \listoffigures
  \fi

  % Tabellenverzeichnis %%%%%%%%%%%%%%%%%%%%%%
  \iflistoftables{}
    \listoftables
  \fi

  \ifglossary{}
    % Abkürzungsverzeichnis %%%%%%%%%%%%%%%%%%%%%%
    % (nur wenn beim usepackage von glossaries-extra die Option "abbreviations" angegeben wurde)
    \makeatletter
    \@ifundefined{printabbreviations}{}{\printabbreviations[title=Abkürzungsverzeichnis]}
    \makeatother

    % sonstiges Glossar
    \printglossary[title=Glossar]
  \fi

  % Formelverzeichnis %%%%%%%%%%%%%%%%%%%%%%
  \iflistofformeln{}
    \listof{Formel}{Formelübersicht}
    \newpage
  \fi
\end{spacing}

\clearpage

\newcounter{romanPagenumber}
\setcounter{romanPagenumber}{\value{page}} % Roemische Seitenanzahl speichern.

% \nocite{*} % enable this if you want to list all your literature, even if it wasn't actually cited in your text

\pagenumbering{arabic}

\begin{spacing}{1.5} % main content gets 1.5 line spacing
  % separate files for each pet with additional details
  \chapter{Ants}

\section{Enclosures}
\begin{figure}[H]
  \centering
  \includegraphics[scale=0.5]{resources/L-niger1.jpg}
  \caption[Lniger1]{enclosure of first Lasius niger colony named ``Inglorious Basterds''}
\end{figure}
\begin{figure}[H]
  \centering
  \includegraphics[scale=0.5]{resources/L-niger2.jpg}
  \caption[L. niger 1]{enclosure of second Lasius niger colony\\named ``Frankenstein's Monsters''}
\end{figure}
\begin{figure}[H]
  \centering
  \includegraphics[scale=0.5]{resources/F-rufibarbis.png}
  \caption[L. niger 1]{enclosure of the Formica rufibarbis colony named ``Foresters''}
\end{figure}


\section{Feeding}
\section{Water}
\section{Catching escapees}
If any ants get out of the enclosure, Don't Panic, just scoop them up with a paint brush and look for where they escaped.
Usually, single escaped ants will want to get back into the enclosure, so they don't venture far from the nest.
If you have a full-scale break-out of hundreds of animals, use the special vacuum accessory pictured below to hoover them up.\todo{add picture}

  \chapter{Argentinische Schaben (Dubia Roaches)}

\section{Säubern}
Gute Nachricht! Das ist nicht nötig!

\section{Füttern}
Gemüse/Früchte im Tiefkühler (Äpfel, Karotten, Pfirsich\ldots{})

\section{Wasser}

\end{spacing}

\clearpage

\pagestyle{plain}


% Literaturverzeichniss - Ab hier wieder Roemische Seitenzahlen

\pagenumbering{roman}
\ifbibliography{}
  \setcounter{page}{\theromanPagenumber}
  %\bibliographystyle{apalike}
  %\bibliography{literatur}
  \printbibliography[title=\prefbiblioname]
  \onehalfspacing{}
  \clearpage
\fi

\pagestyle{empty}
\thispagestyle{empty}

\end{document}
% chktex 17 ignore deliberately unmatched numnber of parentheses (there's a single ')' in the preamble)
