% dummy comment for file-wide intellisense errors

%\usepackage{graphicx} % enables use of eps graphics (encapsulated PostScript). Activate if needed.
%\usepackage{newtx} % replacement of previously used "times" package (using Times font as default)
\usepackage{babel}
\usepackage{supertabular}
\usepackage{wrapfig}
\usepackage{multirow, multicol}
\usepackage[onehalfspacing]{setspace}
\usepackage{scrhack}  % fix float warning of KOMA produced when including listings
\usepackage{listings}
\usepackage{mathptmx}
\usepackage{geometry}
\usepackage{enumitem, amssymb}  % to define todo lists with checkboxes
\usepackage{helvet}
\usepackage{courier}
\usepackage{setspace}
\usepackage{textcomp}
\usepackage[T1]{fontenc}
\usepackage[utf8]{inputenc}
\usepackage{float} % Notwendig fuer figure[h]
\usepackage[german=quotes]{csquotes}
\usepackage[style=alphabetic]{biblatex} % alternative for sort: iso-authoryear
\usepackage{xurl} % better line-breaking than url package. Needs to be added after biblatex to work in bibliography.
\usepackage{pdfpages}
\usepackage{calc} % for calculations with text width
% Fuer Schriftart Arial
%\usepackage[scaled]{uarial}

% Installation der Arial Schriftart unter Linux.
% wget http://tug.org/fonts/getnonfreefonts/install-getnonfreefonts
% texlua install-getnonfreefonts
% getnonfreefonts -r
% getnonfreefonts arial-urw


% PDF Einstellungen für Verlinkungen
\usepackage[
  pdftitle={\Title},
  pdfsubject={\pdfsubject},
  pdfauthor={\Name},
  pdfkeywords={\pdfkeywords}
  hyperfootnotes=false,
  colorlinks=true,
  linkcolor=black,
  urlcolor=black,
  citecolor=black
]{hyperref}

\ifglossary{}
  %%% Abkürzungsverzeichnis (Glossar) Neues Paket (kann nomencl und acronym ersetzen)
  % muss nach hyperref eingebunden werden, um das Paket zu nutzen
  % Abkürzungen werden nur im Glossar angezeigt, wenn sie im Dokument mindestens einmal genutzt wurden
  \usepackage[
    % style=long,
    abbreviations, % Setzt Abkürzungen in ein gesondertes Verzeichnis (nur wenn Glossarverzeichnis auch angezeigt wird)
    % footnote, % Setzt eine Fußnote beim ersten verwendet wird
    % nomain,
    % style=altlist,
  ]{glossaries-extra}
  \setglossarystyle{super}
  \makeglossaries{} % Glossar generieren
\fi

\newfloat{Formel}{H}{for}

%% FAQ environment
\newenvironment{faq}{}{}
\DeclareSectionCommand[
  runin=false,                                        % start the answer in a new line
  afterskip=0.25\baselineskip plus -1ex minus -.2ex,  % chktex 1 commands cannot be terminated with curly braces in arguments
  beforeskip=-2.5ex plus -1ex minus -.2ex,
  indent=0pt,
  level=4,
  font=\usekomafont{paragraph}\itshape, %% using the same font as paragraph, but italic
  tocindent=10em,
  tocnumwidth=5em,
  counterwithin=subsubsection,
  style=section,
]{question}
\newcommand{\faqitem}[2]{\question{#1}{\setlength{\leftskip}{\parindent}#2\par}}
\def\questionautorefname{Frage} % for referring to labels within a question with \autoref, though I prefer referencing with the custom \faqref command
\newcommand*{\faqref}[1]{\hyperref[{#1}]{\appendixautorefname{}~\ref*{#1}, \questionautorefname{}~``\nameref*{#1}''}}


\renewcommand\UrlFont{\color{black}\rmfamily\itshape} % chktex 6 ignore missing '\/' after \itshape

\renewcommand{\familydefault}{\rmdefault}
\newcommand{\bflabel}[1]{\normalfont{\normalsize{#1}}\hfill}

% define list environments for todo checklists
\newlist{todolist}{itemize}{1}
\setlist[todolist]{align=left, leftmargin=*, label=\(\square{}\)}
% additional list with 2 and 3 checkboxes
\newlist{todolistx2}{itemize}{1}
\setlist[todolistx2]{align=left, leftmargin=*, label=\(\square{}\hspace{.75em}\square{}\)}
\newlist{todolistx3}{itemize}{1}
\setlist[todolistx3]{align=left, leftmargin=*, label=\(\square{}\hspace{.75em}\square{}\hspace{.75em}\color{lightgray}\square{}\)}
